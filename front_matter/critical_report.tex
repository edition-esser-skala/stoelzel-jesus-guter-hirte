\documentclass[tocstyle=ref-genre]{ees}

\begin{document}

\eesTitlePage

\eesCriticalReport{
  –     & –    & org    & Bass figures only appear in the following movements (bars in parentheses): 1.3 (15 to 18), 1.5 (32, 39 to 44), 1.7 (8 to 10, 55 to 57), 1.21 (1, 2, 5 to 7, 11, 30), 1.24 (6 to 9, 39, 40), 2.5 (43), 2.15 (17, 35 to 40), 3.4 (4, 6 to 8, 12 to 14, 17), 3.10 (34, 35), 3.17 (1 to 11, 16, 17, 33, 40), 4.4 (41 to 45, 60 to 67), 4.14 (1, 2, 7, 8), and 4.18 (6). The remaining bass figures were added by the editor. \\
  1.12  & –    & –      & Each slur that connects two \eighthNote\ in a group of three \eighthNote\ could also be interpreted as a slur connecting all of these notes. \\
        & 1    & org    & 2nd \halfNote\ in \A1: d4.–\crotchetRest–\quaverRest \\
  1.17  & 5    & vla    & 2nd \quarterNote\ in \A1: \sharp f′8–\sharp f′8 \\
  2.2   & 23   & vl     & 9th \sixteenthNote\ in \A1: g′16 \\
  2.15  & 9    & org    & 2nd \eighthNote\ in \A1: D8 \\
  3.6   & –    & A      & \A1 does not indicate whether A 1 or A 2 should perform this movement. \\
  3.12  & 3    & vl 2   & 1st \halfNote\ in \A1: \sharp c″2 \\
  3.23  & 6    & vla, T & 4th \halfNote\ in \A1: \flat b2 \\
  4.8   & 1, 3 & T      & last \halfNote\ in \A1: a2 \\
  4.9   & 27   & org    & bar in \A1: \flat B1 \\
  A.1f  & –    & –      & These movements appear after 4.22 in \A1. \\
}

\ifPrintFrontMatter
\cleardoublepage
\chapter{Dramatis personæ}

\begin{tabular}{@{} l >\itshape l l l}
      & bewegte      &           & Soprano 1 \\
      & bußfertige   &           & Soprano 1 \\
      & danckbare    &           & Soprano 2 \\
      & demüthige    &           & Soprano 2 \\
      & erschreckte  &           & Soprano 2 \\
  Das & erschrockene & Schäflein & Soprano 1, 2 \\
      & getröstete   &           & Soprano 1 \\
      & glaubende    &           & Soprano 1 \\
      & reuige       &           & Soprano 1 \\
      & verlohrne    &           & Soprano 1 \\
      & wehmüthige   &           & Soprano 2 \\[2ex]
  Das & Evangelium   &           & Alto 1, 2 \\
  Der & Schächer     &           & Alto 2 \\[2ex]
  Der & Evangeliste  &           & Tenore \\
  \textit{JESUS}, & der gute Hirte & & Tenore \\[2ex]
  Das & Gesetze      &           & Basso \\
\end{tabular}


\cleardoublepage
\chapter{Preface}

\textit{Chriſtlicher Leſer!}

Du ſollſt in dieſen Blättern deinen Heyland in ſeinem blutigen Leiden und Sterben ſehen. Dir iſt aber nicht unbekandt, wie es dieſem großen Freunde und theureſten Erlöſer des menſchlichen Geſchlechts je und je gefallen, ſich unter dem Bilde eines \textit{guten Hirten} ſeinen Gläubigen vorzuſtellen. Nicht nur die Bücher des Neuen, ſondern auch des Alten Teſtamentes, ſind voll von dieſen angenehmen Metaphora. Weder mein Vorſatz, noch der Raum leiden es, die Zärtlichkeiten, womit ein treuer Hirte ſeiner geliebten Heerde begegnet, hier weitläufig zu beſchreiben, ſondern ich ſetze nur die Worte Jeſu, Joh. X: \textit{Ein guter Hirte läſſet ſein Leben für die Schafe}, zu weiterer Betrachtung, auch, weil doch von der Liebe eines \textit{guten Hirten} gegen ſeine Heerde gewiß keine größre Probe verlanget, mithin auch nichts Höhers geſagt werden kan.

Und ſiehe! hier lieget \textit{dein guter Hirte}, daß ich ſeines blutigen Schweißes, ſeiner harten Bande, Verläſterungen, Verſpeyungen, Striemen, Beul- und Wunden, ja des ſchmertzlichenTodes am Kreutze geſchweige, hier lieget \textit{dein guter Hirte im Grabe}. O! ſo erſchrick demnach vor dem Donner des \textit{Gesetzes}, wenn du bey ſeinem Tode Berge und Felſen, als erſtaunliche Trauer-Glocken klingen höreſt. Verbinde aber auch dein Hertze mit den Troſt-Specereyen des ſüßen \textit{Evangelii}, wenn du bedenkeſt, daß alle deine Sünden in dem Grabe deines \textit{guten Hirten} verſchloßen seyn. So wirſt du ſeinen Sterbens-Tag, als ein getreues \textit{Schäflein} ihm wohlgefällig, und dir höchſt-erſprießlich begehren, und dermahleinſt mit ihm dich ewig erfreuen.

(from the handwritten libretto)
\fi

\eesToc{
\part{pars1}

\begin{movement}{jesufrommer}
  \voice[Coro]
  Jeſu! frommer Menſchenheerden\\
  guter und getreuer Hirt,\\
  laß mich auch dein Schäflein werden,\\
  das dein Stab und Stimme führt.\\
  Ach! du haſt aus Lieb dein Leben\\
  für die Schafe hingegeben,\\
  und du gabſt es auch für mich,\\
  laß mich wieder lieben dich.
\end{movement}

\begin{movement}{betruebterfall}
  \voice[Das verlohrne Schäflein]
  Betrübter Fall! ach unglückſeelger Bißen!\\
  Wie viel haſt du mir doch entrißen?\\
  Mein Hirte wieß\\
  ein ſchönes Paradieß\\
  mir zur vergnügten Weide an.\\
  Ich war mit Heiligkeit und Unſchuld angethan;\\
  nun aber irr ich in der Wüſten\\
  durch wilde Dorn und Heiden hin.\\
  In meinem böſen Hertzen niſten\\
  verkehrte Neigungen, vergift’ter Wuſt und Grauß,\\
  und brüten lauter Sünden aus.\\
  O weh, daß ich alſo gefallen bin!
\end{movement}

\begin{movement}{achwobin}
  \voice[Das verlohrne Schäflein]
  Ach, wo bin ich hingerathen!\\
  Ach, wie hab ich mich verirrt!\\
  Wehe mir verlaßnen Armen!\\
  Will ſich mein getreuer Hirt\\
  jetzt nicht über mich erbarmen,\\
  ſterb ich in den Mißethaten.
\end{movement}

\clearpage
\begin{movement}{aufsuende}
  \voice[Das Gesetze]
  Auf Sünde folget Fluch und Strafe.\\
  Du wuſteſt deines Hirten Willen,\\
  dir war der rechte Weg bekandt,\\
  doch weil du dich davon gewandt,\\
  ſo muß an dir, boßhaften Schaafe,\\
  ſich Gottes Zorn-Gericht erfüllen.
\end{movement}

\begin{movement}{wereinparadies}
  \voice[Das Gesetze]
  Wer ein Paradieß verſchertzet,\\
  mag auf Dorn und Diſteln gehn.\\
  Schmachte nun auf dürrer Heide,\\
  denn die vorgenoßne Weide\\
  muß vor dich verschloßen ſtehn.
\end{movement}

\begin{movement}{getrostdein}
  \voice[Das\\Evangelium]
  Getroſt! dein Hirte liebt dich noch!\\
  O was verweilſt du doch\\
  auf ſein ſo ſehnliches Begehren\\
  zu ihm zurück zu kehren?\\
  Dein Kyrie Eleiſon\\
  iſt ihm ein angenehmer Thon.\\
  Er richtet das zerſtoßne Rohr\\
  gewiß empor,\\
  und nährt das ſchwache Glaubens-Licht\\
  in der betrübten Seele,\\
  mit ſüßem Freuden-Oehle.\\
  Wer zu ihm kömmt, verſtößt er nicht.
\end{movement}

\begin{movement}{seinhertze}
  \voice[Das\\Evangelium]
  Sein Hertze brennt in ſüßen Liebesflammen\\
  und ſehnet ſich nach dir.\\
  Er ſpricht: Will dich Gesetz und Fluch verdammen,\\
  ſo komme nur zu mir.
\end{movement}

\begin{movement}{eristaufdiesewelt}
  \voice[Der Evangeliste]
  Er iſt auf dieſe Welt gebohren,\\
  zu ſuchen, was verlohren.\\
  Erſt ſucht Er dich im Stall zu Bethlehem,\\
  mit weinenden, doch ſüßen Lippen\\
  rief Er aus Seiner harten Krippen:\\
  Wenn doch mein Schäflein zu mir käm!\\
  Er wolt als Joseph fliehn\\
  und in Egypten ziehn,\\
  damit Er deine Zuflucht würde.\\
  Wie manche ſchwere Bürde\\
  nahm Er in Seinem Lebens Lauf\\
  um deinetwillen willig auf!\\
  Wie oft ließ Er, dich zu gewinnen,\\
  aus Liebe [heiße] Thränen rinnen!\\
  Jetzt, da Er ſterben will,\\
  vermacht Er dir am Ende\\
  Sein Fleiſch und Blut im Neuen Teſtamente.\\
  Ach eile doch in jenen großen Saal\\
  zu Seinem ſüßen Abendmahl.
\end{movement}

\begin{movement}{hungrigdurstig}
  \voice[Coro]
  Hungrig, durſtig und ſehr matt\\
  komm ich armes Schaf gelauffen.\\
  O wie wohl machſt du mich ſatt,\\
  Treuſter Hirt, ſamt deinem Hauffen.\\
  Nicht allein\\
  Brodt und Wein,\\
  ſondern auch dein Leib und Blut\\
  ſind die Weid und edle Hut.
\end{movement}

\begin{movement}{nimmhin}
  \voice[JESUS, der\\gute Hirte]
  Nimm hin und iß, das iſt mein Leib,\\
  komm tränke dich mit meinem Blute.\\
  Ich geb den Leib dem Tode hin\\
  und laß aus meinem treuen Hertzen\\
  bey tauſendfachen Todes Schmertzen\\
  des Blutes Lebens Balsam ziehn,\\
  bloß dir, verlohrnes Schaf, zu gute.
\end{movement}

\begin{movement}{dieliebedeines}
  \voice[Der Evangeliste]
  Die Liebe deines Hirten\\
  will ihre Schäflein nicht allein\\
  in Brod und Wein\\
  mit ſeinem Leib und Blut bewirthen.\\
  Die Demuth will ihn auch\\
  mit einem Schurtz umgürten.\\
  In dieſer mindern Tracht\\
  bückt ſich der HErr der Seraphinen\\
  zu ſeiner Jünger Füßen hin\\
  und ſucht durch diß Bemühn\\
  ſie, und auch den, der auf Verrätherey bedacht,\\
  mit einem Fußbad zu bedienen.
\end{movement}

\clearpage
\begin{movement}{deinexempel}
  \voice[Das demüthige Schäflein]
  Dein Exempel ſoll mir ſtets vor Augen seyn.\\
  Jeſu, gieb daß meine Liebe\\
  ſich an Freund und Feinden übe.\\
  Weyhe du mein Hertz zum Tempel\\
  wahrer Lieb und Demuth ein.
\end{movement}

\begin{movement}{einboeses}
  \voice[Der Evangeliste]
  Ein böſes Schaf von Jesu guter Heerde,\\
  bethört durch ſchändlichen Gewinn,\\
  gibt ihn, daß er getödtet werde,\\
  den Wölffen in den Rachen hin.\\
  Der Hirte weiß, daß die Verrätherey\\
  ein Anfang ſeines Leidens ſey,\\
  und geht mit Danken und mit Beten,\\
  am Oelberg ſolches anzutreten.
\end{movement}

\begin{movement}{keinhirt}
  \voice[Coro]
  Kein Hirt kan ſo fleißig gehen\\
  nach dem Schaf, das ſich verläuft.\\
  Sollſt du Gottes Hertze ſehen,\\
  wie ſich da der Kummer häuft,\\
  wie es dürſtet, lächzt und brennt\\
  nach dem, der ſich abgetrennt\\
  von Ihm und auch von den Seinen,\\
  würdeſt du für Liebe weinen.
\end{movement}

\begin{movement}{verlohrnesschaaf}
  \voice[Das Gesetze]
  Verlohrnes Schaaf, dein guter Hirte\\
  fängt an zu zittern und zu zagen.\\
  Das macht die ungeheure Bürde,\\
  die er will deinetwegen tragen.
\end{movement}

\begin{movement}{gottmuss}
  \voice[Das Gesetze]
  Gott muß durch Blut und Todt verſöhnet ſeyn,\\
  es koſte auch dem Sohne ſelbst das Leben.

  \voice[Das\\Evangelium]
  Und dieſer hat ſich in die Pein\\
  aus Liebe für dich hingegeben.

  \voice[Das Gesetze]
  Du ſollteſt unter ſtetem Weh\\
  ein ewiges Gethſemane\\
  von wegen deiner Sünden\\
  in jenem Schwefel Pfuhl empfinden.

  \voice[Das\\Evangelium]
  Dein Hirte aber reißt dich aus dem Folter Hauß\\
  durch dieſen ſauren Gang herauß.
\end{movement}

\begin{movement}{oliebedienicht}
  \voice[Das bewegte Schäflein]
  O Liebe die nicht zu ermeßen!\\
  Hat Gott denn Gott zu seyn vergeßen,\\
  daß er ſo ängſtlich thut!\\
  Die höchſte Allmacht zagt,\\
  der ewgen Kraft entgeht der Muth.\\
  Es zittert der, auf dem der Himmel ruht.\\
  Mein guter Hirte klagt\\
  in ſeiner höchſten Noth:
\end{movement}

\begin{movement}{meineseele}
  \voice[JESUS, der\\gute Hirte]
  Meine Seele iſt betrübt biß an den Todt.\\
  Wacht und betet, liebſten Schaafe,\\
  weil dem ſichern Sünden Schlafe\\
  ſchreckliche Verſuchung droht.
\end{movement}

\begin{movement}{diesuendenlast}
  \voice[Der Evangeliste]
  Die Sünden Laſt der ganzen Welt\\
  drückt ſeine matten Glieder\\
  zur Erden nieder.\\
  Er kniet, er fältv
  auf das bethränte Angeſicht\\
  und ſpricht:

  \voice[JESUS, der\\gute Hirte]
  Mein Vater! kan es möglich ſeyn,\\
  ſo überhebe mich der ſchweren Pein.\\
  Doch will ich deinen Willen\\
  und nicht den meinigen erfüllen.

  \voice[Der Evangeliste]
  Der einzge Troſt auf dieſer Welt,\\
  den der getreue Hirte\\
  von ſeiner Heerde kunte haben,\\
  liegt in dem tiefſten Schlaf begraben.\\
  Drum öffnet ſich das Himmels Zelt.\\
  Ein Engel muß ihm Stärkung bringen,\\
  damit Er mit dem Tode ringen\\
  und blutgen Schweiß vergießen kan.\\\relax
  [or: und blutge Tropfen ſchwitzen kan.]\\
  So hefftig greifet Gott den Sünden Büßer an!
\end{movement}

\begin{movement}{achsuender}
  \voice[Das reuige Schäflein]
  Ach Sünder!\\
  Nun gehen mir die Augen auf.\\
  Jetz spühr ich deine Heßlichkeit,\\
  da ich ſo vieles Hertzeleid\\
  an meinem Hirten finde.\\
  Mein gantz verirrter Lauf\\
  ging ungeſcheut zur Höllen zu.\\
  Ich ſchlief in fauler Sünden Ruh\\
  und ließe mir auf Sodoms Auen\\
  vor deiner Strafe grauen.\\
  Doch hier ſeh ich erſtaunend an,\\
  wie ſchrecklich Gott die Sünden ſtrafen kan.
\end{movement}

\begin{movement}{achhaettich}
  \voice[Das reuige Schäflein]
  Ach hätt ich euch verfluchte Sünden\\
  doch eh verflucht!\\
  So aber liebt ich meinen Strick\\
  und hielt mein Ungelück für Glück.\\
  Ich ließe mich von dem nicht finden,\\
  der mich geſucht.
\end{movement}

\begin{movement}{sovieljahr}
  \voice[Coro]
  So viel Jahr hab ich gelaufen\\
  den verbothen Irreweg,\\
  und mit dem verfluchten Haufen\\
  ausgeſetzt den guten Steg,\\
  der zur Himmels Pforte führt.\\
  Nie, ach nie hab ich geſpürt\\
  wahre Reue meiner Sünden.\\
  Wo ſoll ich nun Rettung finden?
\end{movement}

\begin{movement}{lassbangefurcht}
  \voice[Das\\Evangelium]
  Laß bange Furcht dich nicht zurücke jagen,\\
  dein guter und getreuer Hirte\\
  hat alles diß getragen,\\
  daß dir geholffen würde.\\
  Sey guthes Muths!\\
  Ein einzger Tropffen Bluts,\\
  der ihm vom Angeſichte fällt,\\
  wiegt mehr als wie die Sünde aller Welt.
\end{movement}

\enlargethispage\baselineskip
\begin{movement}{dieseschoenen}
  \voice[Das\\Evangelium]
  Dieſe ſchönen Blut-Rubinen\\
  werden dir zur Krone dienen\\
  in der frohen Ewigkeit.\\
  Ja, du wirſt mit tauſend Freuden\\
  unter dieſen Roſen weiden,\\
  frey von Sünde, frey von Leid.
\end{movement}

\begin{movement}{waerentausend}
  \voice[Coro]
  Wären tauſend Welt zu finden\\
  von dem Höchſten zugericht,\\
  und du hätteſt alle Sünden,\\
  ſo darinnen ſind, verricht,\\
  wär es doch noch lange nicht\\
  ſo viel, daß das helle Licht\\
  Seiner Gnade hier auf Erden\\
  dadurch könnt erlöſchet werden.

  Mein Gott, öffne mir die Pforten\\
  ſolcher Wohlgewogenheit.\\
  Laß mich allzeit aller Orten\\
  ſchmecken deine Süßigkeit.\\
  Liebe mich und treib mich an,\\
  daß ich dich, ſo gut ich kan,\\
  wiederum umfang und liebe\\
  und ja nun nicht mehr betrübe.
\end{movement}

\part{pars2}

\begin{movement}{nunstellt}
  \voice[Der Evangeliste]
  Nun ſtellt ſich der Verräther ein,\\
  ein Joabs Kuß\\
  ſoll ſeiner Boßheit Deckel ſeyn.\\
  Allein, was dienet diß Verſtellen?\\
  Der gute Hirte fliehet nicht,\\
  ob Er auch für die Schaafe ſterben muß.\\
  Denn nur ein eintzig Wort,\\
  das dieſer Gott Menſch ſpricht,\\
  könnt alle Feinde fällen.\\
  Steck, Petre, nur dein Schwerdt\\
  in ſeine Scheiden,\\
  der Herr, der [jetzo] nicht der Engel Schutz begehrt,\\
  geht gerne an ſein Leiden.\\
  Hör, wie Er muthig ſpricht:
\end{movement}

\begin{movement}{ihrkommtmit}
  \voice[JESUS, der\\gute Hirte]
  Ihr kommt mit Schwerdtern und mit Stangen,\\
  als einen Mörder mich zu fangen,\\
  da ich doch täglich bey euch war.\\
  Doch meines liebſten Vaters willen,\\
  und alle Schrifften zu erfüllen,\\
  reich ich die Hand den Banden dar.
\end{movement}

\clearpage
\begin{movement}{dergrossehirte}
  \voice[Der Evangeliste]
  Der große Hirte wird geſchlagen,\\
  darum zerſtreut die Heerde ſich.\\
  Selbſt Petrus, der ein Felß und Stein\\
  an Muth, Verſtand und Treu wolt ſeyn,\\
  läßt ſich aus Furcht verjagen.
\end{movement}

\begin{movement}{ihrsuenderdenket}
  \voice[Das Gesetze]
  Ihr Sünder, denket nach,\\
  ob eure Boßheit ihres gleichen hat?\\
  Den, welcher noch von blutgen Schweiße triefft,\\
  den Gottes Zorn\\
  an eurer statt\\
  ſo hefftig hat geprüft;\\
  den, der am Oelberg mehr gefühlt,\\
  als wenn ein ſpitzger Dorn,\\
  ein ſcharffer Geißeldrat\\
  den gantzen Leib zerwühlt;\\
  den wollt ihr nun,\\
  da Strick und Bande ihn umfaßen,\\
  alleine laßen?
\end{movement}

\begin{movement}{billigwaer}
  \voice[Das Gesetze]
  Billig wär es, daß euch Gott\\
  wiederum verließe,\\
  daß er euch mit Hohn und Spott\\
  gar von ſich verſtieße.
\end{movement}

\begin{movement}{suchemich}
  \voice[Coro]
  Suche mich, dein Schäflein, wieder,\\
  du mein Gott und treuer Hirt,\\
  welches irrig auf und nieder\\
  Wölffen ſonſt zu Theile wird.\\
  Schließ in Jeſu Wunden ein\\
  das verſcheuchte Täubelein,\\
  daß es Satan nicht erwiſche\\
  in dem wüſten Welt Gebüſche.
\end{movement}

\begin{movement}{dasopfferaller}
  \voice[Der Evangeliste]
  Das Opffer aller Welt\\
  soll erſt der Hoheprieſter ſehn,\\
  drum wird er dieſem vorgeſtellt.\\
  Auch Petrus folget, doch von weiten.\\
  Von dieſem muß der gute Hirte leiden,\\
  daß er auch dreymahl läugnend ſpricht:\\
  Ich kenne dieſes Menſchen nicht.\\
  Doch Jeſus blickt ihn an,\\
  und das Geſchrey von einem Hahn\\
  will dergeſtalt in ſeinen Ohren ſchallen,\\
  daß Buß und Thränen Zeugen seyn,\\
  wie hertzlich er es muß bereun,\\
  daß er ſo ſchwer gefallen.
\end{movement}

\begin{movement}{meinhirtauf}
  \voice[Das bußfertige Schäflein]
  Mein Hirt! auf deßen Treu ich bau,\\
  ich läugne nicht, daß mich die Macht\\
  der Sünden auch zum Fall gebracht.\\
  Doch reut es mich, was ich gethan.\\
  Blick mich, du Gnaden Sonne, an,\\
  zertheil der Seelen Finſternißen,\\
  ſo wird ein milder Thränen Thau\\
  aus Hertz und Augen fließen.
\end{movement}

\begin{movement}{ichbittichruf}
  \voice[Coro]
  Ich bitt, ich ruff, ich weine,\\
  Herr Jeſu, wende dich,\\
  wie Petro mir erſcheine,\\
  und bring zur Ruhe mich.\\
  Ich traue deinem Sterben,\\
  nimm meiner Seel dich an,\\
  ach laß die nicht verderben,\\
  für die du gnug gethan.
\end{movement}

\begin{movement}{mansuchtviel}
  \voice[Der Evangeliste]
  Man ſucht viel falſche Zeugen,\\
  der Unſchuld Recht zu beugen.\\
  Allein,\\
  ihr Zeugniß ſtimmt nicht überein.\\
  Doch auf des Hoheprieſters Fragen,\\
  ob er ein Sohn des Höchſten sey,\\
  will ihm der HErr zur Antwort ſagen:

  \voice[JESUS, der\\gute Hirte]
  Du sagsts,\\
  ich bins.
\end{movement}

\begin{movement}{vonnunan}
  \voice[JESUS, der\\gute Hirte]
  Von nun an ſollt ihr Menſchen ſehen\\
  des Menſchen Sohn zur Rechten ſtehen\\
  der allerhöchſten Macht und Krafft,\\
  wenn er auf denen Wolken Bühnen\\
  in ſeiner Herrlichkeit erſchienen\\
  und rufft die Welt zur Rechenſchafft.
\end{movement}

\begin{movement}{dieswortwird}
  \voice[Der Evangeliste]
  Diß Wort wird auf der Feinde Zungen\\
  zu lauter Gottes Läſterungen\\
  und iſt die Schuld,\\
  weswegen man dem Herrſcher aller Welt\\
  das Todes Urtheil fällt.\\
  Hierauf läßt ſich der HErr\\
  in äußerſter Geduld\\
  verſpotten und verſpeyen.\\
  Man ſchläget und verdeckt\\
  ſein heilges Angeſicht\\
  und will, Er ſoll alſo verſteckt\\
  die Frevler prophezeyen.

  \voice[Das Gesetze]
  Verlohrnes Schaaf, erwäge,\\
  was Gottes Sohn erträgt!\\
  Denck aber auch dabey,\\
  daß deine Schuld es ſey,\\
  die ihn verurtheilt, ſchmäht und ſchlägt.\\
  Weil du den Tod verbrochen,\\
  wird dieſes Urtheil Ihm geſprochen,\\
  und wenn dein Hertze meint,\\
  Er mercke deine Boßheit nicht,\\
  ſo ſchlägſt du ihn in das verdeckte Angeſicht.\\
  Du ſpeyſt Ihm als der ärgſte Feind\\
  auf die vor dich zerrauffte Wangen,\\
  so offt du wider Ihn\\
  auch den geringſten Fehl begangen.

  \voice[Das erschrockene Schäflein]
  Mein Hertz erbebt, wenn es bedencket,\\
  was ich verübt!

  \voice[Das\\Evangelium]
  In dem, der dich biß in den Tod geliebt,\\
  iſt alles dir geſchenket.
\end{movement}

\begin{movement}{wennauchdeiner}
  \voice[Das\\Evangelium]
  Wenn auch deiner Seelen Schade\\
  noch ſo unausſprechlich wär,\\
  doch iſt deines Hirten Gnade\\
  noch weit unausſprechlicher.
\end{movement}

\clearpage
\begin{movement}{wiekanich}
  \voice[Das danckbare Schäflein]
  Wie kan ich Ihm doch danckbar seyn?

  \voice[Das\\Evangelium]
  Bereue deine Schuld,\\
  ergreife ſeine Huld,\\
  und folg Ihm ſtets in Kreutz und Ungemach\\
  als ein getreues Schäflein nach.
\end{movement}

\begin{movement}{guterhirtedieses}
  \voice[Das danckbare Schäflein]
  Guter Hirte, dieſes Hertze\\
  geb ich dir zu eigen hin.\\
  Gib, daß ich zu allen Zeiten,\\
  in den Freuden, in den Leiden,\\
  immer dein Gefehrte bin.
\end{movement}

\begin{movement}{meinhirtich}
  \voice[Coro]
  Mein Hirt, ich bin wohl zufrieden,\\
  wenn du mich nicht von dir ſtößt.\\
  Bleib ich von dir ungeſchieden,\\
  ey, ſo bin ich gnug getröſt.\\
  Laß mich ſeyn dein Eigenthum,\\
  ich verſprech hinwiederum,\\
  hier und dort all mein Vermögen\\
  dir zu Ehren anzulegen.
\end{movement}

\part{pars3}

\begin{movement}{kaumgehtder}
  \voice[Der Evangeliste]
  Kaum geht der Sonnen Licht\\
  am Himmel wieder auf,\\
  ſo bringt der mordbegierge Hauf\\
  den HErrn vors weltliche Gericht.\\
  Indeßen treibt mit ſcharffen Bißen\\
  das ängſtliche Gewißen\\
  den ſchändlichen Verräther an,\\
  zu ſagen: daß er Unrecht hab gethan,\\
  weil er unſchuldig Blut verrathen.\\
  Er hofft ſich zwar, der Angſt\\
  im Tempel zu entladen,\\\relax
  [und] gibt die dreyßig Silberlinge,\\
  ſo er zum Lohn der Ungerechtigkeit empfinge,\\
  verzweifflungsvoll zurück.\\
  Doch drauf erhenckt er ſich an einen Strick.

  \clearpage
  \voice[Das erschrockene Schäflein]
  Verdammter Satans Griff,\\
  wodurch er viele Seelen fället!\\
  Wenn er die Sünden vor der That\\
  als Staub und Sand vorſtellet,\\
  doch, wenn man ſie begangen hat,\\
  zu großen Felſen macht,\\
  die offt dem Glaubens Schiff\\
  betrübt den Untergang gebracht.
\end{movement}

\begin{movement}{acherhoeredoch}
  \voice[Das erschrockene Schäflein]
  Ach erhöre doch mein Sehnen,\\
  du mein Hirte, HErr und Gott!\\
  Wenn auch meines Glaubens Schiff\\
  auf dergleichen Berge lief,\\
  o! ſo mach durch Buß und Thränen\\
  es in Gnaden wieder flott.
\end{movement}

\begin{movement}{derherrder}
  \voice[Der Evangeliste]
  Der HErr, der ſchon zuvor geſagt,\\
  Er würde unter Macht der Heyden\\
  den Tod des Kreutzes müßen leiden,\\
  wird peinlich vor Pilato angeklagt:\\
  daß Er des Volckes Treu\\
  vom Kayſer ab und auf ſich wende,\\
  indem Er ſelbſt geſtände,\\
  daß Er ein König sey.\\
  Pilatus foderte von ihm\\
  das Zeugniß ſelber ab,\\
  worauf Er ihm zur Antwort gab:
\end{movement}

\begin{movement}{meinreichist}
  \voice[JESUS, der\\gute Hirte]
  Mein Reich iſt nicht von dieſer Welt.\\
  Ich bin als König zwar gekommen,\\
  daß ich die Wahrheit zeugen ſolt,\\
  doch Iſrael hat nicht gewolt.\\
  Mein Hauß hat mich nicht aufgenommen,\\
  weil Sünd und Greul darinnen Hoffſtadt hält.
\end{movement}

\begin{movement}{pilatusfindet}
  \voice[Der Evangeliste]
  Pilatus findet keine Schuld an ihm.\\
  Jedoch der Prieſter Ungeſtüm\\
  fährt weiter noch mit Klagen fort.\\
  Der Heyland aber ſagt kein eintzges Wort.
\end{movement}

\clearpage
\begin{movement}{dasschafverstummt}
  \voice[Das\\Evangelium]
  Das Schaf verſtummt vor ſeinem Scheerer,\\
  die Unſchuld ſchweigt in Sanfftmuth ſtill,\\
  und der von Gott geſandte Lehrer\\
  redt nicht mehr, da Er ſterben will.\\
  Hiermit gibt Er dir zu verſtehen,\\
  du Schäflein ſolſt nur unverzagt\\
  auf deines Hirten Lippen ſehen,\\
  wenn Sünd und Hölle dich verklagt.
\end{movement}

\begin{movement}{rededurchdein}
  \voice[Coro]
  Rede durch dein Stilleſchweigen,\\
  liebſter Jeſu, mir das Wort,\\
  wenn mich Sünden überzeugen\\
  und der Klagen fort und fort,\\
  wenn mein böß Gewißen ſchweyget\\
  und mir die Verdamniß dräuet.\\
  Ach, laß deine Todes Pein\\
  nicht an mir verlohren seyn!
\end{movement}

\begin{movement}{derunschuldsonnen}
  \voice[Der Evangeliste]
  Der Unschuld Sonnen Licht\\
  ſoll in dem Marter Kreyſe wandern\\
  von einem Unthier zu dem andern.\\
  Drum bringt man Jeſum nun\\
  vor des Herodes Hoffgericht.\\
  Doch ſchweyget Er auf viel Befragen\\
  und muß zuletz ein weißes Kleid\\
  zum Schimpf zurücke tragen.
\end{movement}

\begin{movement}{diesistdaskleid}
  \voice[Das Gesetze]
  Diß iſt das Kleid,\\
  das Satan dir geraubt,\\
  damit, daß du dem Höchſten gleich zu seyn geglaubt.\\
  Erſchrick, verlohrnes Schaaf,\\
  vor der Gerechtigkeit,\\
  die Gott allhier ergehen läßt!\\
  Und glaube feſt,\\
  du ſeyſt in deinem ſündlichen Gewand\\
  aus Gottes Angeſicht verbannt.
\end{movement}

\begin{movement}{erstauntihrnicht}
  \voice[Das Gesetze]
  Erſtaunt ihr nicht, befleckte Sünder,\\
  vor Gottes Strafgerechtigkeit?\\
  Wird Gottes Sohn im weißen Kleide\\
  ein Spott und Greul verruchter Leute,\\
  was meinet ihr boßhafften Kinder,\\
  das ihr zu dulden würdig seyd?
\end{movement}

\begin{movement}{herodesundpilatus}
  \voice[Der Evangeliste]
  Herodes und Pilatus ſehen,\\
  daß Jeſu ſey zu viel geſchehen,\\
  drum zeigt der Letztere\\
  den neidiſchen Verklägnern\\
  des HErren Unſchuld Sonnenklar,\\
  will auf das Oſterfeſt\\
  nach der Gewohnheit leben,\\
  und ſtatt des Barrabä,\\
  der ein gefangner Mörder war,\\
  Ihm ſeine Freyheit wiedergeben;\\
  des ſie ſich aber doch aus Boßheit wegern.\\
  Sie ſuchen Barrabam,\\
  den Wolff, den Mörder, zu befreyn\\
  und wollen über das unſchuldge Lamm\\
  das Crucifige ſchreyn.
\end{movement}

\begin{movement}{kommherund}
  \voice[Das Evanglium]
  Komm her und ſiehe deinen Hirten,\\
  verzagtes Schäflein, näher an.\\
  Diß eben, daß der Feinde Liſt\\
  gantz unvermögend iſt,\\
  ihm einges Unrecht aufzubürden,\\
  diß iſts, was dich erfreuen kan.\\
  Denn leidet Gott, der keine Schulden hat,\\
  an deiner ſtatt,\\
  ſo kanſt du nun des feſten Glaubens leben:\\
  In Jeſu ſey dir alle Schuld vergeben.
\end{movement}

\begin{movement}{diesenhonigsuessen}
  \voice[Das\\Evangelium]
  Dieſen honigsüßen Spruch\\
  ſchreib dir in das Hertze.\\
  Er iſt eine Freudenkertze\\
  in der trüben Sünden Nacht.\\
  Er vertreibt den Höllen Schmertze,\\
  welchen das Gewißen macht,\\
  unter des Gesetzes Fluch.
\end{movement}

\clearpage
\begin{movement}{eysorichte}
  \voice[Coro]
  Ey, ſo richte dich empor,\\
  du betrübtes Angeſicht!\\
  Laß das Seuffzen, nimm hervor\\
  deines Glaubens Freuden Licht,\\
  diß behalt, wenn dich die Nacht\\
  deines Kummers traurig macht.
\end{movement}

\begin{movement}{nachdemdergeissel}
  \voice[Der Evangeliste]
  Nachdem der Geißel ſcharffer Zahn\\
  den heilgen Leib zerrißen,\\
  legt Ihm der rasendtolle Hauff\\
  den Purpur Mantel an\\
  und ſetzt Ihm eine Krone\\
  von ſpitzgen Dornen auf,\\
  gibt Ihm ein Rohr in ſeine rechte Hand,\\
  fällt doch aus Spott und Hohne\\
  zu ſeinen Füßen,\\
  fängt an, als König Ihn zu grüßen,\\
  speyt aber auch zugleich Ihm in das Angeſicht\\
  und ſchont des blutgen Haupts mit harten Schlägen nicht.\\
  Nachdem er alſo zugericht\\
  und überall von Blut und Speichel naß,\\
  führt ihn Pilatus ſelbſt heraus, und ſpricht:\\
  Seht! welch ein Menſch iſt das!
\end{movement}

\begin{movement}{bespiegelteuch}
  \voice[Das Gesetze]
  Beſpiegelt euch, geſchminckte Pfauen,\\
  in dieſem blutgen Jammer Bild!\\
  O bleibt ihr hier vor Furcht und Grauen,\\
  von Angſt und Schrecken unerfüllt,\\
  ſo müßt ihr härter noch als Stein\\
  und wilder als ein Heyde ſeyn.\\
  Ach Sünder, ach erwäge!\\
  So viel Wunden, Beul und Schläge,\\
  ſo viel Verſpotten und Verſpeyen\\
  zahlt Jeſus jetzt dem Richter dar,\\
  dich vom Gerichte zu befreyen,\\
  das deines Hochmuths Straffe war.
\end{movement}

\begin{movement}{dubistvon}
  \voice[Das Gesetze]
  Du biſt von Erden, Staub und Aſchen,\\
  was prahlſt und prangerſt du ſo ſehr?\\
  Ach wolleſt du dies wohl befleißen,\\
  in Gottes Augen ſchön zu heißen,\\
  ſo müßteſt du je mehr und mehr\\
  dich in der Buße Thränen waſchen.
\end{movement}

\begin{movement}{diefeindefahren}
  \voice[Der Evangeliste]
  Die Feinde fahren immer fort,\\
  den Richter zu der Unſchuld Mord\\
  mit Schreyen zu bewegen.\\
  Jedoch Pilatus ſetzt dagegen:\\
  Es ſey nichts ſtrafbahres an Ihm zu finden.\\
  So ſoll die Klage ſich nunmehr auf dieſes gründen:\\
  Es habe ihr Geſetz den Tod Ihm zugedacht,\\
  weil Er ſich ſelbſt zu Gottes Sohn gemacht.\\
  Pilatus aber will hierauf\\
  zur Antwort Ihn verbinden,\\
  mit Vorwand, daß ſein Tod\\
  und Leben bey ihm ſtünden.\\
  Drum thut Er ſeinen Mund mit dieſen Worten auf:
\end{movement}

\begin{movement}{dubesaessest}
  \voice[JESUS, der\\gute Hirte]
  Du beſäßeſt keine Macht\\
  über meinen Tod und Leben,\\
  wär ſie nicht in jener Nacht\\
  dir von oben her gegeben.\\
  Doch die Kläger habens Schuld,\\
  und ich leide mit Gedult.
\end{movement}

\begin{movement}{pilatuswillsofort}
  \voice[Der Evangeliste]
  Pilatus will ſofort\\
  ſich eyfriger beſtreben,\\
  die Freyheit Ihm zu geben.\\
  Jedoch der Feinde Wort\\
  bringt ihm die ſchnöde Meynung bey:\\
  daß er, gäb er den Klagen kein Gehör,\\
  des Kayſers Freund nicht ſey,\\
  weil Jeſus ſich für einen König hielt.\\
  Sobald er Menſchenfurcht im Hertzen fühlt,\\
  bricht er den Todes Stab,\\
  nimmt Waßer, wäſcht die Hände ab,\\
  und will an Jeſu Blut und Pein\\
  hiemit unſchuldig ſeyn.\\
  Allein das Volck lädt freventlich\\
  diß unſchuldige Blut auf ſich.\\
  Den Barrabam ſtellt man auf freyen Fuß,\\
  da Jeſus noch vor ſeinem Tode\\
  die ſcharffe Geißel leiden muß.
\end{movement}

\begin{movement}{istsnichtgenug}
  \voice[Das wehmüthige Schäflein]
  Iſts nicht genug, daß mein getreuer Hirte\\
  ſein Todes Urtheil angehört?\\
  Wird Ihm die harte Jammer Bürde\\
  durch Blut und Wunden noch vermehrt!\\
  O mehr als Tygerhaffte Wuth!\\
  Die Mordbegier\\
  erſäufft ſich ſonſt im warmen Blut,\\
  doch hier,\\
  da Jeſu Blut wie Ströhme fließet,\\
  thut es nichts mehr,\\
  als daß es Oel ins Feuer gießet.\\
  Ach edler Leib! darum zerfleiſcht man dich ſo ſehr\\
  und läßt dich faſt von allem Blute leer,\\
  weil mein verderbtes Fleiſch und Blut\\
  gar niemahls gutes thut.
\end{movement}

\begin{movement}{wennmichnach}
  \voice[Das wehmüthige Schäflein]
  Wenn mich nach den tollen Freuden\\
  dieſer eitlen [or: der verrückten] Welt gelüſt,\\
  ach [or: o] ſo zeige mir dein Leiden,\\
  das gantz [or: welches] unausſprechlich iſt.\\
  Stell mir vor, wie du im Garten\\
  blutgefärbten Schweiß vergoßt,\\
  und auf wie viel bittre Arten\\
  du den Kreutzes Kelch gekoſt.\\
  Laß mich ſehn, wie Strick und Bande\\
  dich unſchuldges Lamm umfaßt,\\
  und wie viele Schmach und Schande\\
  du für mich erduldet haſt.\\
  Zeige mir, wie aus den Beulen\\
  ſich dein rothes Blut ergoß,\\
  als dich an der Marter Säulen\\
  Geißel, Peitſch und Ruth umſchloß.
\end{movement}

\begin{movement}{lassdeineliebe}
  \voice[Coro]
  Laß deine Liebe doch\\
  in Marck und Seele dringen,\\
  hilff mir, mein ſündlich Fleiſch\\
  mit allen Lüſten zwingen.\\
  Halt meine Seele rein,\\
  mein Hertze sey dein Haus\\
  und wirfft, was weltlich iſt,\\
  durch deinen Geiſt heraus.
\end{movement}

\clearpage
\part{pars4}

\begin{movement}{seelegehauf}
  \voice[Coro]
  Seele, geh auf Golgatha,\\
  ſetz dich unter Jeſus Kreutze,\\
  und bedencke, was dich da\\
  für ein Trieb zur Buße reitze.\\
  Willſt du unempfindlich ſeyn,\\
  o, ſo biſt du mehr als Stein.
\end{movement}

\begin{movement}{daskreutzdaran}
  \voice[Der Evangeliste]
  Das Kreutz, daran man Jeſum hefftet,\\
  trägt Er, biß auf den Todt entkräfftet,\\
  auf den verwundten Rücken fort,\\
  biß Simon von Cyrenen,\\
  der von dem Felde kam,\\
  es, doch gezwungen, übernahm.\\
  Viel Weiber folgten Ihm mit Thränen\\
  biß zum beſtimmten Ort,\\
  Ihn zu beklagen, nach.\\
  Zu dieſen wandt Er ſein Geſicht\\
  und ſprach:
\end{movement}

\begin{movement}{ihrtoechterjerusalems}
  \voice[JESUS, der\\gute Hirte]
  Ihr Töchter von Jeruſalem,\\
  beweint mich nicht,\\
  weint über euch und eure Kinder.\\
  Denn ja, man wird bey künfftgen Tagen\\
  mit Furcht und Zittern ſagen:\\
  Glückſelig iſt der Leib,\\
  der keine Frucht gezeugt!\\
  Glückſelig iſt das Weib,\\
  das nie geſäugt!\\
  Da werden ſich die Sünder\\
  aus Bangigkeit und Schrecken\\
  mit Felſen ſuchen zu bedecken.\\
  Drum, will man diß am grünen Holtz begehn,\\
  was wird man ſich am dürren unterſtehn?
\end{movement}

\begin{movement}{soschrecklichhier}
  \voice[Das Gesetze]
  So ſchrecklich hier die Strafe iſt,\\
  ſo greulich iſt vor Gott die Sünde.\\
  Gott weiſet dir, o Sünden Knecht,\\
  an ſeinem eingebohrnen Kinde,\\
  wie hoch ſein allerheilgſtes Recht\\
  dein frevelhafftes Thun empfinde,\\
  weil es ſein Sohn ſo ſchmertzlich büßt.
\end{movement}

\begin{movement}{kaumalser}
  \voice[Der Evangeliste]
  Kaum als Er zu der Schedelſtätte\\
  in großer Mattigkeit gelanget war,\\
  bot man, als ob man Mitleyd hätte,\\
  Ihm Myrrhen Wein\\
  nebſt Gall und Eßig dar.\\
  Doch als Ers ſchmeckt,\\
  wolt Ers nicht trincken.\\
  Drauf wurd Er an das Kreutz\\
  und beyderſeits\\
  zur Rechten und zur Lincken,\\
  zwey Übelthäter angepflöckt,\\
  Pilati Schrifft dabey,\\
  daß Er der Juden König ſey,\\
  zu oberſt an das Kreutz geſteckt.
\end{movement}

\begin{movement}{erschrocknesschaeflein}
  \voice[Das\\Evangelium]
  Erſchrocknes Schäflein, komm nach Golgatha!\\
  Schau, deines Hirten Arme\\
  ſind liebreich ausgeſpannt,\\
  daß Er ſich dein erbarme\\
  und dir das Paradieß aufſchließe,\\
  daraus du warſt verbannt.\\
  Sein Kreutz wird dir zum Baum des Lebens,\\
  an ſolchem blüht die Frucht,\\
  die deiner Kehlen ſüße,\\
  und die du nicht vergebens\\
  ſo ſehnlich haſt geſucht.
\end{movement}

\begin{movement}{kommsetzedich}
  \voice[Das\\Evangelium]
  Komm, ſetze dich in Schatten,\\
  des du begehret haſt.\\
  Entſchütte dich der Laſt,\\
  laß hier mit dir ſich gatten\\
  vergnügte Ruh und Raſt.
\end{movement}

\enlargethispage\baselineskip
\begin{movement}{ruhnurauf}
  \voice[Coro]
  Ruh nur auf Jeſu liebſten Hertzen\\
  als ſein gefundnes Schäflein ſtill,\\
  ſo er, nach vieler Dornen Schmertzen,\\
  an ſeiner Brust erwärmen will,\\
  und trägt dich heim zur rechten Spur\\
  der andern Schäflein, ruhe nur.
\end{movement}

\begin{movement}{derheylandhinge}
  \voice[Der Evangeliste]
  Der Heyland hinge nackt und bloß,\\
  und über ſeinem Kleide\\
  warff ſchon das Krieges Volck das Loß,\\
  als Er die hochbetrübten Beyde,\\
  Mariam und Johannem, ſah,\\
  zu welchen dieſes Wort vom Kreutz herab geſchah:

  \voice[JESUS, der\\gute Hirte]
  Weib, ſiehe, dieſer iſt dein Sohn,\\
  und du ſolſt ſeine Mutter ſeyn.

  \voice[Der Evangeliste]
  Und die fürüber gingen,\\
  belegten Ihn bey aller ſeiner Pein\\
  mit Läſterungen, Spott und Hohn,\\
  des gleichen auch, die bey Ihm hingen.\\
  Biß der zur rechten Hand\\
  des Hirten Unſchuld doch erkant\\
  und in bußfertger Seelen Angſt\\
  als ein verlohrnes Schäflein ſich\\
  mit dieſem Glaubens Wort zu Ihm gewandt:

  \voice[Der Schächer]
  Gedenck, o HErr, an mich,\\
  wenn du in deinem Reich anlangſt.

  \voice[Der Evangeliste]
  Worauf der HErre ſich vernehmen ließ:

  \voice[JESUS, der\\gute Hirte]
  Ich ſage dir, glaub du es ſicherlich,\\
  heut wartet noch auf dich\\
  das ſchöne Paradieß.
\end{movement}

\begin{movement}{olippendie}
  \voice[Das glaubende Schäflein]
  O Lippen! die von Honig triefen.\\
  O Mund! der aus dem Tod\\
  ins Leben rufen kan.\\
  Nun hör ich keinen Teufel an,\\
  wie ſehr er auch mit Höllenflammen droht.\\
  Braußt immer hin,\\
  ihr fürchterlichen Tieffen [or: Rieſen],\\
  eröffne, Abgrund, deinen Rachen;\\
  da ich bey meinem Hirten bin,\\
  kan ich die größte Noth,\\
  die ſonst die Sünde macht, verlachen.
\end{movement}

\begin{movement}{achvondiesem}
  \voice[Das glaubende Schäflein]
  Ach von dieſem Kreutze thaut\\
  lauter Troſt auf mein Gewißen.\\
  Sollt ich nun verzagen müßen,\\
  da mein Glaube dem vertraut,\\
  der nach einen Schächer ſchaut,\\
  ihm den Himmel aufzuſchließen?
\end{movement}

\begin{movement}{esfreutschon}
  \voice[Coro]
  Es freut ſchon Jeſus ſich,\\
  daß Er ſein Schäflein, dich\\
  wird auf die Achſeln legen\\
  und dich auf guten Wegen\\
  zu ſeiner Heerde bringen,\\
  die wird für Freuden ſingen.
\end{movement}

\begin{movement}{undumdie}
  \voice[Der Evangeliste]
  Und um die ſechſte Stunde\\
  ward eine dicke Finſterniß.\\
  Dabey vernahm man diß\\
  aus Jeſu Munde:

  \voice[JESUS, der\\gute Hirte]
  Mein Gott! Mein Gott!\\
  Warum haſtu mich verlaßen?

  \voice[Der Evangeliste]
  Die Feinde trieben ihren Spott\\
  gewohntermaßen,\\
  ſogar mit Jeſu Angstgeſchrey.\\
  Er aber rief:

  \voice[JESUS, der\\gute Hirte]
  Mich dürſtet.

  \voice[Der Evangeliste]
  Und einer von den Knechten lief,\\
  nahm einen Schwamm,\\
  ſteckt ſolchen auf ein Rohr,\\
  und hielt dem faſt erwürgten Lamm\\
  auf ſolche Weiſe Ysopen und Eßig vor.\\
  Hierauf wird Jeſus laut:

  \voice[JESUS, der\\gute Hirte]
  Es iſt vollbracht.

  \voice[Der Evangeliste]
  Und abermahl ſchrie Er mit gantzer Macht:

  \voice[JESUS, der\\gute Hirte]
  Mein Vater, ich befehle meinen Geiſt\\
  in deine Hände.

  \voice[Der Evangeliste]
  Und als Er nun gezahlet und geleiſt,\\
  was Er doch nicht geraubt,\\
  ſo neigte ſich mit ſeinem blutgen Haupt\\
  ſein Leben auch zum Ende.
\end{movement}

\begin{movement}{ihrhimmelklagt}
  \voice[Coro]
  Ihr Himmel, klagt,\\
  ihr Wolcken, gießet Thränen!\\
  Denn Gott iſt todt.\\
  O großes Wort!\\
  Gott wird gequälet.\\
  O großer Mord!\\
  Gott wird entſeelet.\\
  O niemals noch erhörte Noth!\\
  Wer iſt, der nicht erſtaunt und ſagt:\\
  Ihr Himmel, klagt,\\
  ihr Wolcken, gießet Thränen!\\
  Denn Gott iſt todt.
\end{movement}

\begin{movement}{jedochwasregen}
  \voice[Das erschreckte Schäflein]
  Jedoch was regen ſich bey dieſem Todt\\
  für ſchreckensvolle Trauer Glocken?\\
  O weh! ach unerhörte Noth!\\
  Mein Troſt fängt in mir an zu ſtocken.\\
  O Furcht! o Bangigkeit!\\
  Der Himmel geht in einem Trauer Kleid\\
  und hüllet ſeinen Schein\\
  in ſchwartze Trauer Decken ein.\\
  Es bebet Hertz und Geiſt,\\
  denn ja, die Erde zittert unter mir\\
  und ſcheinet zu zerfallen.\\
  Ich ſeh, wie dort und hier\\
  bei fürchterlichen Knallen\\
  ein Felß zerſpringt, ein Berg zerreißt [or: zerſpringt].\\
  Wohin, ach Gott! wohin\\
  ſoll ich erschrecktes Schäflein fliehn?\\
  Der Himmel ſcheint durch ſeinen Blick mich zu verjagen,\\
  die Erde bebt, mich ferner nicht zu tragen.
\end{movement}

\begin{movement}{ausdernatur}
  \voice[Das Gesetze]
  Aus der Natur erzürntem Blicke\\
  erkenne deine Mißethat.\\
  Dem Himmel graut dich anzuſehen,\\
  drum will ſein Licht verhüllet ſtehen.\\
  Der Abgrund bebt, die Felſen ſpringen\\
  und ſind bereit, dich zu verſchlingen.\\
  Warum? weil deine Sünd und Tücke\\
  den Schöpffer jetz getödtet hat.
\end{movement}

\clearpage
\begin{movement}{fliehenicht}
  \voice[Das\\Evangelium]
  Erschrecktes Schäflein, fliehe nicht!\\
  Zeigt die Natur dir ihr erzürnt Geſicht,\\
  ſo ſiehe über dich\\
  nach Jeſu zugeſchloßnen Augen,\\
  die werden dir zum Troſte taugen.\\
  Erbebt der feſte Grund der Erden\\
  und ſpringen Felſen auch entzwey,\\
  laß dir darum nicht bange werden,\\
  du biſt in deines Hirten Wunden\\
  vor allen Unfall frey.\\
  Da ſchließe dich mit feſten Glauben ein\\
  und ruhe nach der Pein,\\
  die du zuvor empfunden.
\end{movement}

\begin{movement}{druecketeuch}
  \voice[Coro]
  Drücket euch an meine Lippen,\\
  o, ihr werthen Wunden ihr!\\
  Roſenrothe Purpur Klippen,\\
  oeffnet eure Höhlen mir.\\
  Thaut mir Gnade, Heyl und Leben,\\
  hertzet euch mit meinem Geiſt.\\
  Eure Fülle kan mir geben\\
  alles, was Vergnügung heißt.
\end{movement}

\begin{movement}{derabendbrach}
  \voice[Der Evangeliste]
  Der Abend brach nun an,\\
  als Joſeph, der dem HErren zugethan,\\
  um den erblaßten Leichnam bate.\\
  Pilatus wundernsvoll, daß er bereits\\
  verſtorben wäre,\\
  gab dieſer Bitte gern Gehöre.\\
  Und alſo nahm\\
  er Ihn vom Kreutz.\\
  Auch Nicodemus kam,\\
  der vormahls in der Nacht bey Jeſu war,\\
  und brachte Specereyen dar.\\
  Mit dieſen bunden ſie\\
  des HErrn entſeelte Glieder\\
  in reine Leichen Tücher ein,\\
  verdeckten drauf des Grabes Höhle\\
  mit einen großen Stein.
\end{movement}

\clearpage
\begin{movement}{sohatdich}
  \voice[Das getröstete Schäflein]
  So hat dich denn der Liebe Macht,\\
  o Hirt und Biſchoff meiner Seele,\\
  ans Kreutz zum Todt, ja biß ins Grab gebracht.\\
  Ach, daß ich doch mit tauſend Zungen redte!\\
  zu preiſen, was du mir erwieſen.\\
  Ach, daß ich doch gnug Waßers hätte,\\
  in einer See von Liebes Thränen zu zerfließen.\\
  Doch weil ich weiß,\\
  daß ich für deine Liebe\\
  dir keinen größern Danck kan geben,\\
  als wenn ich mich befleiß,\\
  zu Ehren dir zu leben.\\
  So ſollen jetzt aus danckbeſeelten Triebe\\
  auf deinem Leichen Stein\\
  dir meine Sinnen aufgeopffert ſeyn.
\end{movement}

\begin{movement}{nimmhirte}
  \voice[Coro]
  Nimm, Hirte, mein Geſicht durch deine Schönheit ein,\\
  daß meine Augen blind zum Augen Lüſten ſeyn!\\
  Erfülle mein Gehör mit deinem Angſtgeſchrey,\\
  daß dieſe Stimme mir ſtets vor den Ohren ſey!\\
  Dein bittrer Gallen Tranck verbittre mir die Welt,\\
  wenn etwa dem Geſchmack ihr Sodoms Obſt gefällt!\\
  Hilff, edle Saarons Blum, daß immer mein Geruch\\
  an deiner Lieblichkeit ſich zu ergetzen ſuch!\\
  Durchdringe mein Gefühl mit deinem guten Geiſt\\
  biß es zum böſen Todt zum Guten lebend heißt!
\end{movement}

\begin{movement}{diesalles}
  \voice[Coro]
  Diß alles, obs für ſchlecht zwar iſt zu ſchätzen,\\
  wirſt du es doch nicht gar bey Seite ſetzen.\\
  In Gnaden wirſt du diß von mir annehmen,\\
  mich nicht beſchämen.

  Wenn dort, HErr Jeſu, wird vor deinem Throne\\
  auf meinem Haupte ſtehn die Ehrenkrone,\\
  da will ich dir, wenn alles wird wohl klingen,\\
  Lob und Danck ſingen.
\end{movement}

\clearpage
\part{appendix}

\begin{movement}{ergabsogar}
  \voice[Soprano, Alto, Tenore]
  Er gab ſogar ſein eigen Leben\\
  für die verlohren Schaafe hin.\\
  Halt ich mich nur zu ſeiner Weide,\\
  ſo weiß ich, daß ich allem Leide\\
  durch ſeinen Schutz entnommen bin.
\end{movement}

\begin{movement}{diesallesossia}
  \voice[Coro]
  Diß alles, obs für ſchlecht zwar iſt zu ſchätzen,\\
  wirſt du es doch nicht gar bey Seite ſetzen.\\
  In Gnaden wirſt du diß von mir annehmen,\\
  mich nicht beſchämen.

  Wenn dort, HErr Jeſu, wird vor deinem Throne\\
  auf meinem Haupte ſtehn die Ehrenkrone,\\
  da will ich dir, wenn alles wird wohl klingen,\\
  Lob und Danck ſingen.
\end{movement}
}

\eesScore

\end{document}
